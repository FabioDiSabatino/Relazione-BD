\chapter{UDF}
\label{udf}
\section{UDF del NMC}
\label{udfnmc}
\justify
In questa sezione verranno mostrate tutte le User Defined Function (da qui in avanti indicate con l'acronimo UDF) che implementano NMC e che quindi calcolano del valore di exposure delle stazione ferroviarie.\\ Il linguaggio utilizzato nelle UDF è PL/pgSQL offerto da PostgreSQL, in particolare sono state fruttatele una serie di funzioni fornite dall’estensione spaziale PostGIS.\\
La struttura utilizzata per l'utilizzo delle UDF consiste in nell'esecuzione di una funzione principale denominata \textit{esposizione} che richiama 3 UDF. Queste vengono eseguite consecutivamente in quanto l'elaborazione dell'una prevede l'utilizzo dei risultati dell'elaborazione della precedente. \\
All'interno dell'UDF \textit{esposizione} per ogni stazione analizzata vengono richiamate nell'ordine le UDF \textit{quotaMediaZoneVicine}, \textit{quotaStazione} e \textit{dislivelloPiuVicino}. Le 4 UDF citate vanno a implementare i 4 algoritmi descritti nella sezione \ref{algoritmonmc}. In Particolare \textit{quotaMediaZoneVicine} implementa \textit{averageElevationNearZones} , \textit{quotaStazione} implementa \textit{elevationBuild}, \textit{dislivelloPiuVicino} implementa \textit{slopeFactor} ed infine \textit{esposizione} implementa \textit{computeExposure}.\\
Poiché lo scopo di questa sezione è mostrare come sono state implementate le cose e non quello di spiegarne la logica di funzionamento, all'interno delle UDF per semplicità realizzativa è stato utilizzato un naming diverso rispetto a quello utilizzato precedentemente. Per questo motivo prima di ogni UDF verrà mostrata una tabella riassuntiva che mostra come questi nomi sono stati mappati.   

\subsection{quotaMediaZoneVicine}
La funzione calcola la quota delle zone che intersecano il buffer attraverso la media delle \textit{elevation} delle linee di livello che intersecano ognuna di queste zone. Inoltre la funzione conserva nella tabella \textit{zoneVicine} tutte le zone che intersecano il buffer con la quota media calcolata in mo da consentire alle successive UDF un immediato utilizzo di questi risultati.
\begin{itemize}
\item \textbf{input} \textit{buffer GEOMETRY} rappresenta la geometria del buffer circolare centrato sulla stazione in esame.
\item \textbf{output} \textit{VOID}
\end{itemize} 

\begin{table}[]
\centering
\caption{tabella di mappatura dei nomi}
\label{mapTb1}
\begin{tabular}{|c|c|}
\hline
Schema logico & UDF      \\ \hline
Zones         & zone     \\
Isoipse       & lineelvl \\ \hline
\end{tabular}
\end{table}

\newpage
\begin{lstlisting}[style=mySQL]
CREATE or REPLACE FUNCTION quotaMediazoneVicine(buffer GEOMETRY) RETURNS VOID
	LANGUAGE plpgsql
	as $$
	DECLARE
    -- variabili di ausilio per il calcolo --
		zona RECORD;
		linea RECORD;
		quotamedia FLOAT;
		contatore INT;
	BEGIN
    -- inizializzazione delle variabili per il calcolo della quota media delle zone--
		quotamedia:=0;
		contatore:=0;
    -- ciclo su tutte le zone della geoarea che intersecano il buffer --
		FOR zona IN (SELECT * FROM zone WHERE st_intersects(zona.geom, buffer)) LOOP
      -- ciclo su tutte le linee di livello che intersecano la zona in esame --
      FOR linea IN (SELECT * FROM lineelvl WHERE st_intersects(zona.geom,linea.geom)) LOOP
        -- somma della quota delle linee di livello --
        quotamedia:=quotamedia+linea.elevation;
        contatore:=contatore+1;
      END LOOP;
      IF contatore>0 THEN
        quotamedia=quotamedia/contatore;
      END IF;
      INSERT INTO zoneVicine(geom, quotamedia, szk)VALUES (zona.geom, quotamedia, zona.szk);
      quotamedia:=0;
      contatore:=0;
		END LOOP;
	END;
$$;
\end{lstlisting}