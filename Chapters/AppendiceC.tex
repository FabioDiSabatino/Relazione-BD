\justify
\chapter{User Defined Function}
\label{udf}
\section{UDF del NMC}
\label{udfnmc}
In questa sezione verranno mostrate tutte le UDF (da qui in avanti indicate con l'acronimo UDF) che implementano NMC e che quindi calcolano del valore di exposure delle stazione ferroviarie.\\ Il linguaggio utilizzato nelle UDF è PL/pgSQL offerto da PostgreSQL, in particolare sono state fruttatele una serie di funzioni fornite dall’estensione spaziale PostGIS.\\
La struttura utilizzata per l'esecuzione delle UDF consiste in una funzione principale denominata \textit{esposizione} che richiama le altre UDF. Queste vengono eseguite consecutivamente in quanto l'elaborazione dell'una prevede l'utilizzo dei risultati dell'elaborazione della precedente. \\
All'interno dell'UDF \textit{esposizione} per ogni stazione analizzata vengono richiamate nell'ordine le UDF \textit{quotaMediaZoneVicine}, \textit{stimaQuotaStazione} e \textit{dislivelloPiuVicino}. Le 4 UDF citate vanno a implementare i 4 algoritmi descritti nella sezione \ref{algoritmonmc}. In Particolare \textit{quotaMediaZoneVicine} implementa \textit{averageElevationNearZones} , \textit{stimaQuotaStazione} implementa \textit{elevationBuild}, \textit{dislivelloPiuVicino} implementa \textit{slopeFactor} ed infine \textit{esposizione} implementa \textit{computeExposure}.\\
Poiché lo scopo di questa sezione è mostrare come sono state implementate le cose e non quello di spiegarne la logica di funzionamento, all'interno delle UDF per semplicità realizzativa è stato utilizzato un naming diverso rispetto a quello utilizzato precedentemente. Per questo motivo prima di ogni UDF verrà mostrata una tabella di corrispondenza dei nomi.\\
L'invocazione alla main function \textit{esposizione} eseguita durante la sperimentazione è la seguente:

\begin{lstlisting}[style=mySQL]
SELECT esposizione('stazioni', FALSE, 5, 500);
\end{lstlisting}   

Dai parametri della funzione si osserva che il fattore \textit{alpha\_slope} è settato a 5, mentre il raggio del buffer è posto pari a 500 m.

\subsection{\textbf{ESPOSIZIONE}}
Rappresenta la main function che richiama le altre UDF. \newline
\textbf{Input:}
\begin{itemize}
\item \textit{tabInput text} corrisponde al nome della tabella contenente i punti sul quale si vuole calcolare l'exposure. 
\item \textit{stessaTab BOOL} corrisponde a un booleano che se posto a \textit{TRUE} permette di salvare i risultati nella stessa tabella di input altrimenti li conserva nella tabella di default \textit{tabella\_risultati}. I parametri dell'UDF \textit{tabInput} e \textit{stessaTab} sono stati inseriti per dare un maggior grado di libertà al codice in modo da poter essere riutilizzato come verrà fatto in seguito dal NMCE. 
\item \textit{alphaSlope FLOAT} rappresenta il parametro che da un peso al fattore pendenza.
\item \textit{raggioBuffer FLOAT } rappresenta il raggio del buffer circolare centrato sulla stazione espresso in \textit{m}.
\end{itemize}
\textbf{Output:} \textit{VOID} 

\begin{table}[h]
\centering
\caption{Tabella di corrispondenza dei nomi per l'UDF esposizione}
\label{mapTb3}
\begin{tabular}{|l|l|}
\hline
Schema logico       & UDF                \\ \hline
elevation           & quota              \\
slope               & pendenza           \\
experiment\_station & tabella\_risultati \\ \hline
\end{tabular}
\end{table}

\begin{lstlisting}[style=mySQL]
CREATE or REPLACE FUNCTION esposizione(tabInput text, stessaTab BOOL, alphaSlope FLOAT, raggioBuffer FLOAT) RETURNS VOID
	LANGUAGE plpgsql
	as $$
	DECLARE
		-- variabili di ausilio per il calcolo --
		stazione RECORD;
		raggio FLOAT;
		buffer geometry;
		idLineaPiuVicina INT;
		fattorePendenzaPiuVicina FLOAT;
		areaVicina RECORD;
		percentualeCopertura FLOAT;
		exp FLOAT;
		pesoPendenza FLOAT;
		quotaArray INT [2];
		quotaStimataStazione INT;
	BEGIN
		pesoPendenza:=alphaSlope; -- pesopendenza rappresenta alphaSlope
		raggio:=raggioBuffer; -- parametro che indica il raggio del buffer circolare centrato sulla stazione --
		-- creazione della tabella dei risultati --
		IF NOT(stessaTab) THEN
			CREATE TABLE IF NOT EXISTS tabella_risultati(id SERIAL PRIMARY KEY, gidStazione INT REFERENCES stazioni(gid), quota INT, pendenza FLOAT, exposure FLOAT) ;
		END IF;
		-- pulizia della tabella per evitare risultati doppi --
		DELETE FROM tabella_risultati;
		-- creazione delle tabelle di appoggio tempo, tempo1 e zoneVicine --
		CREATE TABLE zoneVicine( id SERIAL  PRIMARY KEY, geom geometry(MULTIPOLYGON), quotamedia FLOAT, szk FLOAT);
		CREATE TABLE tempo(id SERIAL PRIMARY KEY, geom geometry(MULTILINESTRING), elevation INT);
		CREATE TABLE tempo1(id INT PRIMARY KEY, geom geometry(MULTILINESTRING),elevation INT, distanza FLOAT);

		-- ciclo su tutte le stazioni --
		FOR stazione IN  EXECUTE 'SELECT * FROM '|| tabInput ||''  LOOP
			exp:=0;
			buffer:=st_buffer(stazione.geom, raggio);

			-- chiamate alle udf che vanno a calcolare i coefficienti necessari per il calcolo dell'exposure --
			PERFORM quotaMediaZoneVicine(buffer);
			quotaArray:=(SELECT stimaQuotaStazione(stazione.gid, stazione.geom));
			idLineaPiuVicina:=quotaArray[1];
			quotaStimataStazione:=quotaArray[0];
			fattorePendenzaPiuVicina:=(SELECT dislivelloPiuVicino(buffer, raggio, stazione.geom, quotaStimataStazione, idLineaPiuVicina));

		-- ciclo sulle zone che che intersecano il buffer --
		FOR areaVicina IN (SELECT * FROM zoneVicine) LOOP
				percentualeCopertura:=0;
				percentualeCopertura:=(SELECT st_area(st_intersection(areaVicina.geom,buffer))/st_area(areaVicina.geom));
				IF st_intersects(stazione.geom,areaVicina.geom) OR st_distance(stazione.geom,areaVicina.geom)<1
				THEN
					exp:=exp+percentualeCopertura*areaVicina.szk*((@(quotaStimataStazione-areaVicina.quotamedia)));
				ELSE
					exp=exp+(percentualeCopertura*areaVicina.szk*((@(quotaStimataStazione-areaVicina.quotamedia)))/(st_distance(stazione.geom,areaVicina.geom))^1);
				END IF;
			END LOOP;
			exp:=exp+(fattorePendenzaPiuVicina*pesoPendenza);
			IF stessaTab THEN
				EXECUTE ( 'UPDATE '|| tabInput ||' SET exposure='|| exp ||' WHERE '|| tabInput ||'.gid='|| stazione.gid ||'');
			ELSE
				INSERT INTO tabella_risultati(gidStazione, quota, pendenza, exposure) VALUES (stazione.gid, quotaStimataStazione, (fattorePendenzaPiuVicina*pesoPendenza), exp);
			END IF;
		END LOOP;

		-- cancellazione tabelle di appoggio --
		DROP TABLE zoneVicine;
		DROP TABLE tempo;
		DROP TABLE tempo1;
	END;
$$;
\end{lstlisting}

\subsection{\textbf{QUOTAMEDIAZONEVICINE}}
La funzione calcola la quota delle zone che intersecano il buffer attraverso la media delle \textit{elevation} delle linee di livello che intersecano ognuna di queste zone. Inoltre la funzione conserva nella tabella \textit{zoneVicine} tutte le zone che intersecano il buffer con la quota media calcolata in modo da consentire alle successive UDF un immediato utilizzo di questi risultati.\\
\textbf{Input:} 
\begin{itemize}
\item \textit{buffer GEOMETRY} rappresenta la geometria del buffer circolare centrato sulla stazione in esame.
\end{itemize}
\textbf{Output:} \textit{VOID} 

\begin{table}[h]
\centering
\caption{Tabella di corrispondenza dei nomi per l'UDF quotaMediaZoneVicine}
\label{mapTb1}
\begin{tabular}{|c|c|}
\hline
Schema logico & UDF      \\ \hline
Zones         & zone     \\
Isoipse       & lineelvl \\ \hline
\end{tabular}
\end{table}

\begin{lstlisting}[style=mySQL]
CREATE or REPLACE FUNCTION quotaMediaZoneVicine(buffer GEOMETRY) RETURNS VOID
	LANGUAGE plpgsql
	as $$
	DECLARE
		-- variabili di ausilio per il calcolo --
		zona RECORD;
		linea RECORD;
		quotamedia FLOAT;
		contatore INT;
	BEGIN
		-- inizializzazione delle variabili per il calcolo della quota media delle zone --
		quotamedia:=0;
		contatore:=0;
		-- pulizia della tabella di appoggio --
		DELETE FROM zoneVicine;
		-- ciclo su tutte le partizioni della geoarea che intersecano il buffer --
		FOR zona IN (SELECT * FROM zone) LOOP
			IF(st_intersects(zona.geom, buffer)) THEN
				-- ciclo su tutte le linee di livello che intersecano la zona in esame --
				FOR linea IN (SELECT * FROM lineelvl) LOOP
					IF st_intersects(zona.geom,linea.geom) THEN
						-- somma della quota delle linee di livello --
						quotamedia:=quotamedia+linea.elevation;
						contatore:=contatore+1;
					END IF;
				END LOOP;
				IF contatore>0 THEN
					quotamedia=quotamedia/contatore;
				END IF;
				INSERT INTO zoneVicine(geom, quotamedia, szk)VALUES (zona.geom, quotamedia, zona.szk);
				quotamedia:=0;
				contatore:=0;
			END IF;
		END LOOP;
	END;
$$;
\end{lstlisting}

\subsection{\textbf{STIMAQUOTASTAZIONE}}
La funzione calcola una stima della quota della stazione attraverso una media ponderata tra l'\textit{elavation} delle linee di livello più vicine ad essa. Inoltre la funzione conserva nella tabella \textit{tempo} tutte le intersezione tra le linee di livello e le zone che intersecano il buffer centrato sulla stazione, in modo da consentire alle successive UDF un immediato utilizzo di questi risultati.\\
\textbf{Input:} 
\begin{itemize}
\item \textit{gidStazione INTEGER} rappresenta l'identificativo della stazione in esame.
\item \textit{geomStazione GEOMETRY} rappresenta la geometria della stazione in esame.
\end{itemize}
\textbf{Output:} \textit{INT[]} la funziona ritorna un array di interi di due elementi dove il primo elemento corrisponde alla quota stimata della stazione, mentre il secondo contiene l'identificativo della linea più vicina all'interno della tabella \textit{tempo}.

\begin{table}[h]
\centering
\caption{Tabella di corrispondenza dei nomi per l'UDF stimaQuotaStazione}
\label{mapTb2}
\begin{tabular}{|l|l|}
\hline
Schema logico & UDF      \\ \hline
Isoipse       & lineelvl \\ \hline
\end{tabular}
\end{table} 

\begin{lstlisting}[style=mySQL]
CREATE or REPLACE FUNCTION stimaQuotaStazione(gidStazione INTEGER, geomStazione GEOMETRY) RETURNS INT[]
	LANGUAGE plpgsql
	as $$
	DECLARE
		-- variabili di ausilio per il calcolo --
		areaVicina RECORD;
		linea RECORD;
		lineaInZonaVicina RECORD;
		dmin FLOAT;
		dquasimin FLOAT;
		idmin INT;
		idquasimin INT;
		dist FLOAT;
		altezzaStazione INT;
		risultatiFun INT [2];
	BEGIN
		-- pulizia della tabella --
		DELETE FROM tempo;
		-- ciclo sulle zone che intersecano il buffer centrato sulla stazione --
		FOR areaVicina IN (SELECT * FROM zoneVicine) LOOP
		-- ciclo sulle linee di livello che intersecano la zona in esame --
			FOR linea IN (SELECT * FROM lineelvl) LOOP
				IF st_intersects(areaVicina.geom,linea.geom) THEN
					-- inserimento nella tabella tempo l'intersezione tra la zona e la linea, con la quota associata alla linea --
					INSERT INTO tempo( geom, elevation) VALUES((SELECT st_multi(st_intersection(areaVicina.geom,linea.geom))), linea.elevation);
				END IF;
			END LOOP;
		END LOOP;
		-- dmin conterrà il valore della distanza tra la linea di livello più vicina e la stazione --
		dmin:=10000;        
		-- dquasimin conterrà il valore della distanza tra la seconda linea di livello più vicina e la stazione --
		dquasimin:=10000;   
		-- idmin conterrà l'identificativo della linea a distanza minore dalla stazione --
		idmin:=0;           
		-- idquasimin conterrà l'identificativo della seconda linea a distanza minore dalla stazione --
		idquasimin:=0;      

		-- ciclo sui segmenti di linea che intersecano le zone che intersecano con il buffer centrato sulla stazione --
		FOR lineaInZonaVicina IN (SELECT * FROM tempo) LOOP
			dist:=st_distance(geomStazione, lineaInZonaVicina.geom);
			IF dist<=dmin THEN
				dquasimin:=dmin;
				idquasimin:=idmin;
				dmin:=dist;
				idmin:=lineaInZonaVicina.id;
			END IF;
			IF dist>dmin AND dist<=dquasimin THEN
				dquasimin:=dist;
				idquasimin:=lineaInZonaVicina.id;
			END IF;
		END LOOP;
		IF(st_intersects(st_makeline(st_closestpoint((SELECT tempo.geom FROM tempo WHERE id=idquasimin),(geomStazione)),geomStazione),(SELECT tempo.geom FROM tempo WHERE tempo.id=idmin))) THEN
			dquasimin:=dmin;
			idquasimin:=idmin;
		END IF;
		-- calcolo della quota della stazione come media ponderata delle quote delle linee di livello più vicina ad essa --
		altezzaStazione:=((SELECT tempo.elevation FROM tempo WHERE id=idmin)*dquasimin+(SELECT tempo.elevation FROM tempo WHERE id=idquasimin)*dmin)/(dmin+dquasimin);
		risultatiFun[0]=altezzaStazione;
		risultatiFun[1]=idmin;
		RETURN  risultatiFun;
	END;
$$;
\end{lstlisting}

\subsection{\textbf{DISLIVELLOPIUVICINO}}
La funzione calcola la somma delle pendenze tra le linee di livello lungo il vettore direzionale.\\
\textbf{Input:} 
\begin{itemize}
\item \textit{buffer GEOMETRY} rappresenta la geometria del buffer circolare centrato sulla stazione in esame.
\item \textit{raggioBuffer FLOAT} rappresenta il raggio del buffer circolare espresso in \textit{m}.
\item \textit{geomStazione GEOMETRY} rappresenta la geometria della stazione in esame.
\item \textit{quotaStazione INT} rappresenta la quota stimata della stazione in esame.
\item \textit{idLineaPiuVicina INT} rappresenta l'identificativo della linea di livello più vicina all'interno della tabella \textit{tempo}.
\end{itemize}
\textbf{Output:} \textit{FLOAT} la funzione restituisce la somma delle pendenze calcolata. 

\begin{table}[h]
\centering
\caption{Tabella di corrispondenza dei nomi per l'UDF dislivelloPiuVicino}
\label{mapTb3}
\begin{tabular}{|l|l|}
\hline
Schema logico & UDF      \\ \hline
Isoipse       & lineelvl \\ \hline
\end{tabular}
\end{table} 

\begin{lstlisting}[style=mySQL]
CREATE or REPLACE FUNCTION dislivelloPiuVicino(buffer GEOMETRY, raggioBuffer FLOAT, geomStazione GEOMETRY, quotaStazione INT, idLineaPiuVicina INT ) RETURNS FLOAT
	LANGUAGE plpgsql
	as $$
	DECLARE
		-- variabili di ausilio per il calcolo --
		linea RECORD;
		puntoVicino GEOMETRY;
		azimuth FLOAT;
		lunghezzaLinea FLOAT;
		vettoreDirezionale GEOMETRY;
		lineaInBuffer RECORD;
		primo BOOL;
		monotonia BOOL;
		salita BOOL;
		stazioneStessaQuotaLinea BOOL;
		sommaPendenza FLOAT;
		geomUltimaLinea GEOMETRY;
		elevationUltimaLinea INT;
	BEGIN
		-- pulizia della tabella di appoggio --
		DELETE FROM tempo1;
		-- ciclo sulle linee di livello che intersano il buffer centrato sulla stazione --
		FOR linea IN (SELECT * FROM lineelvl) LOOP
			IF st_intersects(buffer,linea.geom) THEN
			-- inserimento in una tabella dell'intersezione tra la linea e il buffer, con la quota associata e l'id --
			INSERT INTO tempo1(id, geom, elevation) SELECT linea.gid, (SELECT st_multi(st_intersection(buffer,linea.geom))), linea.elevation WHERE NOT exists(SELECT tempo1.id FROM tempo1 WHERE tempo1.id=linea.gid);
			END IF;
		END LOOP;
		 -- costruzione del vettore direzionale --
		puntoVicino:=st_closestpoint((SELECT tempo.geom FROM tempo WHERE id=idLineaPiuVicina),(geomStazione));
		azimuth:=st_azimuth(geomStazione,puntoVicino);
		lunghezzaLinea:=raggioBuffer+1;
		vettoreDirezionale:=st_makeline(geomStazione, st_translate(geomStazione, sin(azimuth)*lunghezzaLinea, cos(azimuth)*lunghezzaLinea));

		-- ciclo sui segmenti di linee di livello che intersecano il buffer --
		FOR lineaInBuffer IN(SELECT * FROM tempo1)LOOP
			IF st_intersects(lineaInBuffer.geom, vettoreDirezionale) THEN
				-- aggiornamento delle ennumple in tempo1 con la distanza con la stazione calcolata lungo il vettore direzionale --
				UPDATE tempo1 SET distanza=(SELECT st_distance(geomStazione, st_intersection(lineaInBuffer.geom, vettoreDirezionale))) WHERE tempo1.id=lineaInBuffer.id;
			ELSE
				DELETE FROM tempo1 WHERE tempo1.id=lineaInBuffer.id;
			END IF;
		END LOOP;
		-- inizializzazione di variabili booleane utilizzate per interrompere l'algoritmo --
		primo:=TRUE;
		monotonia:=TRUE;
		salita:=FALSE;
		stazioneStessaQuotaLinea:=FALSE;
		sommaPendenza:=0;
		FOR lineaInBuffer IN(SELECT * FROM tempo1 ORDER BY distanza)LOOP
			IF (monotonia=TRUE) THEN

				-- calcola della pendenza tra la stazione e la prima linea di livello incontrata lungo il vettore direziona --
				IF (primo=TRUE) THEN
					sommaPendenza:=sommaPendenza+((@(lineaInBuffer.elevation-quotaStazione))/(st_distance(geomStazione, lineaInBuffer.geom)));
					IF lineaInBuffer.elevation>quotaStazione THEN
						salita:=TRUE;
					END IF;
					IF lineaInBuffer.elevation=quotaStazione THEN
						stazioneStessaQuotaLinea:=TRUE;
					END IF;
					primo:=FALSE ;

				-- calcola della pendenza tra una linea di livello e la successiva  --
				ELSE
					IF (stazioneStessaQuotaLinea=FALSE) THEN
						IF ((salita=TRUE) AND lineaInBuffer.elevation>elevationUltimaLinea) OR ((salita=FALSE) AND lineaInBuffer.elevation<elevationUltimaLinea) THEN
							sommaPendenza:=sommaPendenza+(@(lineaInBuffer.elevation-elevationUltimaLinea))/st_distance(st_intersection(geomUltimaLinea, vettoreDirezionale), st_intersection(lineaInBuffer.geom, vettoreDirezionale));
						ELSE
							monotonia:=FALSE ;
						END IF;
					ELSE
						IF lineaInBuffer.elevation>elevationUltimaLinea THEN
							salita:=TRUE ;
						END IF;
						sommaPendenza:=sommaPendenza+(@(lineaInBuffer.elevation-elevationUltimaLinea))/st_distance(st_intersection(geomUltimaLinea, vettoreDirezionale), st_intersection(lineaInBuffer.geom, vettoreDirezionale));
						stazioneStessaQuotaLinea:=FALSE;
					END IF;
				END IF;

				geomUltimaLinea:=lineaInBuffer.geom;
				elevationUltimaLinea:=lineaInBuffer.elevation;
			END IF;
		END LOOP;
		RETURN sommaPendenza;
	END;
$$;
\end{lstlisting}


\section{User Defined Function del NMCE}
\label{udfnmce}
In questa sezione verranno mostrate tutte le UDF che implementano NMCE e che contribuiscono al calcolano valore di exposure delle linee ferroviarie.\\ 
Così come per le stazioni ferroviarie la struttura utilizzata per l'esecuzione delle UDF relative alle linee ferroviarie consiste in una funzione principale denominata \textit{esposizione\_linee} che richiama le altre UDF.
Queste vengono eseguite consecutivamente in quanto l'elaborazione dell'una prevede l'utilizzo dei risultati dell'elaborazione della precedente. \\
All'interno dell'UDF \textit{esposizione\_linee} vengono richiamate nell'ordine le UDF \textit{semplificaLinee}, \textit{campiona}, \textit{dividiInSegmenti}, \textit{esposizione} e \textit{expSegmenti}.
Poiché lo scopo di questa sezione è mostrare come sono state implementate le cose e non quello di spiegarne la logica di funzionamento, all'interno delle UDF per semplicità realizzativa è stato utilizzato un naming diverso rispetto a quello utilizzato precedentemente. Per questo motivo prima di ogni UDF verrà mostrata una tabella riassuntiva che mostra come questi nomi sono stati mappati. 
L'invocazione alla main function \textit{esposizione\_linee} eseguita durante la sperimentazione è la seguente:

\begin{lstlisting}[style=mySQL]
SELECT esposizione_linee(300, 1000);
\end{lstlisting}     

\subsection{\textbf{ESPOSIZIONE\_LINEE}}
Rappresenta la main function che richiama le altre UDF.\\
\textbf{Input:} 
\begin{itemize}
\item \textit{passoCampionamento INTEGER} definisce il passo di campionamento espresso in \textit{m} delle linee ferroviarie.
\item \textit{lunghezzaSegmento INTEGER} definisce la lunghezza che si vuole dare ai segmenti che formano le linee ferroviarie.
\end{itemize}
\textbf{Output:} \textit{VOID} 

\begin{lstlisting}[style=mySQL]
CREATE or REPLACE FUNCTION esposizione_linee(passoCampionamento INT, lunghezzaSegmento INT) RETURNS VOID
	LANGUAGE plpgsql
	as $$
	DECLARE
	BEGIN
		-- creazione tabella di appoggio --
		CREATE TABLE only_line(id SERIAL PRIMARY KEY, old_id INT, geom geometry(LINESTRING), p INTEGER[]);
		-- creazione tabelle dei risultati --
		CREATE TABLE IF NOT EXISTS punti_linea(gid SERIAL PRIMARY KEY, old_id INT REFERENCES linee(gid), exposure FLOAT, geom geometry(POINT));
		CREATE TABLE IF NOT EXISTS segmenti( id SERIAL PRIMARY KEY, numero SERIAL, old_id INT REFERENCES linee(gid), geom geometry(LINESTRING), exp_segmento FLOAT);
		-- pulizia delle tabelle --
		DELETE FROM punti_linea;
		DELETE FROM segmenti;
		DELETE FROM only_line;
		
		PERFORM semplificaLinee();
		PERFORM campiona(passoCampionamento);
		PERFORM dividiInSegmenti(lunghezzaSegmento);
		-- richiamo della main function del NMC
		PERFORM esposizione('punti_linea', TRUE, 5, 500);
		PERFORM expSegmenti();
		-- cancellazione tabella di appoggio --
		DROP TABLE only_line;
	END;
$$;
\end{lstlisting}

\subsection{\textbf{SEMPLIFICALINEE}}
La funzione a partire dalle \textit{MultiLineString} che rappresentano le linee ferroviarie ricava delle geometrie più semplici ossia \textit{Linestring} e le salva nella tabella \textit{only\_line}.\\

\begin{table}[h]
\centering
\caption{Tabella di corrispondenza dei nomi per l'UDF semplificaLinee}
\label{mapTb4}
\begin{tabular}{|l|l|}
\hline
Schema logico       & UDF                \\ \hline
Raylway          & linee              \\ \hline
\end{tabular}
\end{table}

\begin{lstlisting}[style=mySQL]
CREATE or REPLACE FUNCTION semplificaLinee() RETURNS VOID
	LANGUAGE plpgsql
	as $$
	DECLARE
		-- variabili di ausilio --
		linea RECORD;
		lineaSemplice GEOMETRY;
	BEGIN
		ALTER SEQUENCE only_line_id_seq RESTART WITH 1;
		-- ciclo sulle linee ferroviarie --
		FOR linea in (SELECT * FROM linee) LOOP
			lineaSemplice:=(SELECT st_linemerge(linea.geom));
			-- si tenta di riunire multilinestring in linestring, se non è possibile si spezza la multilinestring in più linestring --
			IF st_geometrytype(lineaSemplice)='ST_MultiLineString' THEN
				INSERT INTO only_line(geom, old_id, p) VALUES ((st_dump(st_linemerge(linea.geom))).geom, linea.gid, (st_dump(st_linemerge(linea.geom))).path) ;
			ELSE
				INSERT INTO only_line(geom, old_id) VALUES (lineaSemplice, linea.gid);
			END IF;
		END LOOP;
	END;
$$;
\end{lstlisting}

\subsection{\textbf{CAMPIONA}}
Funzione che campiona la le linee semplici presenti nella tabella \textit{only\_line} contente le \textit{Linestring} ricavate dalle linee ferroviarie.\\
\textbf{Input:} 
\begin{itemize}
\item \textit{passo INTEGER} definisce il passo di campionamento.
\end{itemize}
\textbf{Output:} \textit{VOID} 

\begin{table}[h]
\centering
\caption{Tabella di corrispondenza dei nomi per l'UDF campiona}
\label{mapTb5}
\begin{tabular}{|l|l|}
\hline
Schema logico       & UDF                \\ \hline
Sampling\_point          & punti\_linea              \\ \hline
\end{tabular}
\end{table}

\begin{lstlisting}[style=mySQL]
CREATE or REPLACE FUNCTION campiona(passo INT) RETURNS VOID
	LANGUAGE plpgsql
	as $$
	DECLARE
		-- variabili di ausilio --
		linea RECORD;
		percentuale FLOAT;
		lunghezzaLinea FLOAT;
		lineaPrecedente INT;
		i INT;
	BEGIN
		ALTER SEQUENCE punti_linea_gid_seq RESTART WITH 1;
		lineaPrecedente:=0;
		-- ciclo sulle linee semplici ricavate dalle linee ferroviarie --
		FOR linea in (SELECT * FROM only_line ORDER BY id) LOOP
			lunghezzaLinea:=st_length(linea.geom);
			percentuale:=(passo*1)/lunghezzaLinea;
			i:=1;
			-- ciclo fino a che con il campionamento non si supera la lunghezza della linestring --
			WHILE (passo*i)<lunghezzaLinea LOOP
				INSERT INTO punti_linea(old_id, geom) VALUES (linea.old_id, (SELECT st_lineinterpolatepoint(linea.geom, percentuale*i)));
				i:=i+1;
			END LOOP;
		END LOOP;
	END;
$$;
\end{lstlisting}

\subsection{dividiInSegmenti}
Funzione che divide la le linee semplici, presenti nella tabella \textit{only\_line} contente le \textit{Linestring} ricavate dalle linee ferroviarie, in segmenti di lunghezza fissa.\\
\textbf{Input:} 
\begin{itemize}
\item \textit{passo INTEGER} definisce il la lunghezza dei segmenti che si vuole ottenere.
\end{itemize}
\textbf{Output:} \textit{VOID} 

\begin{table}[h]
\centering
\caption{Tabella di corrispondenza dei nomi per l'UDF dividiInSegmenti}
\label{mapTb6}
\begin{tabular}{|l|l|}
\hline
Schema logico       & UDF                \\ \hline
Experiment\_railway          & segmenti              \\ \hline
\end{tabular}
\end{table}

\begin{lstlisting}[style=mySQL]
CREATE or REPLACE FUNCTION dividiInSegmenti(passo INT) RETURNS VOID
	LANGUAGE plpgsql
	as $$
	DECLARE
		-- variabili di ausilio --
		linea RECORD;
		lineaPrecedente INT;
		lunghezzaLinea FLOAT;
		percentuale FLOAT;
		i INT;
		startFraction FLOAT;
		endFraction FLOAT;
	BEGIN
		passo:=1000;
		lineaPrecedente:=0;
		ALTER SEQUENCE segmenti_id_seq RESTART WITH 1;
		-- ciclo sulle linee semplici ricate dalle linne ferroviarie --
		FOR linea IN (SELECT * FROM only_line ORDER BY old_id, p DESC ) LOOP
			-- numero viene riazzerato quando si comincia la divisione di una nuova linea --
			IF lineaPrecedente!=linea.old_id THEN
				ALTER SEQUENCE segmenti_numero_seq RESTART WITH 1;
			END IF;
			lineaPrecedente:=linea.old_id;
			lunghezzaLinea:=st_length(linea.geom);
			percentuale:=(passo*1)/lunghezzaLinea;
			i:=1;
			-- ciclo fino al superamento della lunghezza della linea --
			WHILE (passo*i)<lunghezzaLinea LOOP
				startFraction:=percentuale*(i-1);
				endFraction:=(percentuale*(i-1))+percentuale;
				INSERT INTO segmenti(old_id, geom) VALUES (linea.old_id,  ST_LineSubstring(linea.geom, startFraction, endFraction));
				i:=i+1;
			END LOOP;
			-- controlli per l'ultimo segmento della linea --
			IF (passo*i)>=lunghezzaLinea AND passo*(i-1)!=lunghezzaLinea AND percentuale<=1 THEN
				INSERT INTO segmenti(old_id, geom) VALUES (linea.old_id, ST_LineSubstring(linea.geom, endFraction, 1));
			END IF;
			IF percentuale>1 THEN
				INSERT INTO segmenti(old_id, geom) VALUES (linea.old_id,  ST_LineSubstring(linea.geom, 0, 1));
			END IF;
		END LOOP;
	END;
$$;
\end{lstlisting}

\subsection{\textbf{EXPSEGMENTI}}
Funzione che per ogni segmento delle linee ferroviarie calcola un valore di \textit{exposure} pari alla media del valore di \textit{exposure} dei punti campionati che interseca.\\

\begin{table}[h]
\centering
\caption{Tabella di corrispondenza dei nomi per l'UDF dividiInSegmenti}
\label{mapTb6}
\begin{tabular}{|l|l|}
\hline
Schema logico       & UDF                \\ \hline
Experiment\_railway          & segmenti              \\ 
Sampling\_point          & punti\_linea              \\ \hline
\end{tabular}
\end{table}

\begin{lstlisting}[style=mySQL]
CREATE or REPLACE FUNCTION expSegmenti() RETURNS VOID
	LANGUAGE plpgsql
	as $$
	DECLARE
		-- variabili di ausilio --
		segmento RECORD;
		punto RECORD;
		exp FLOAT;
		contatore INT;
	BEGIN
		-- ciclo su tutti i segmenti --
		FOR segmento IN (SELECT * FROM segmenti) LOOP
			contatore:=0;
			exp:=0;
			-- ciclo su tutti i punti della stassa linea del segmento --
			FOR punto in (SELECT * FROM punti_linea WHERE punti_linea.old_id=segmento.old_id) LOOP
				IF (st_distance(segmento.geom, punto.geom)<1) THEN
					exp:=exp+punto.exposure;
					contatore:=contatore+1;
				END IF;
			END LOOP;
			IF contatore!=0 THEN
				exp:=exp/contatore;
			END IF;
			UPDATE segmenti SET exp_segmento=exp WHERE segmenti.id=segmento.id;
		END LOOP;
	END;
$$;
\end{lstlisting}